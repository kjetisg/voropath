\documentclass[dissertation.tex]{subfiles}
\begin{document}
\chapter{Parametric curves}
\section{Splines}\label{sec:spline}
A \emph{spline} $\mathbf{S}(t)$ is a parametric function - defined in a certain space
(bidimensional in this work) as a
function of a certain parameter $t$ - that is composed from polynomial
functions, piecewise in intervals in $t$.

Formally we define the parametric domain
$$[a,b]\subset\mR$$
and a partition of that space defined by the \emph{nodes}
$$\tau = \{\tau_0,\dots,\tau_l\}$$
such that $a=\tau_0<\tau_1<\dots<\tau_{l-1}<\tau_l=b$ forming $l$
intervals
$$
I_i=
\begin{cases}
  [\tau_i,\tau_{i+1}) & \mbox{if } i=0,\dots,l-2\\
    [\tau_i,\tau_{i+1}] & \mbox{if } i=l-1\\
\end{cases}
$$
is possible to define the following spaces:
\paragraph{Piecewise polynomial functions space} $P_{m,\tau}$
is the space of the functions that are polynomials of maximum degree $m$
in each interval $I_i$ of the partition, formally:
\begin{multline*}
  P_{m,\tau}=\{f:[a,b]\rightarrow\mR\ \mid\ \exists p_0\dots
  p_{l-1}\in\Pi_m \ \text{such that}\\
  f(t)=p(t),\ \forall t\in I_i,\
  i=0\dots l-1\}
\end{multline*}
where $\Pi_m$ is the space of the polynomials of degree from $0$ to
$m$. The dimension of this space is $l\cdot(m+1)$ because have $l$
polynomials and the dimension of $\Pi_m$ is $m+1$.
\paragraph{Classic spline functions space} $S_{m,\tau}$ is the space of
the piecewise polynomial functions that have continuity $C^{m-1}$ in
the junctions of the intervals, formally:
$$
S_{m,\tau}=P_{m,\tau}\cap C^{m-1}[a,b].
$$
The dimension of this space is $l\cdot(m+1)-(l-1)\cdot m\,=\,l+m$.
\paragraph{Generalized spline functions space} $S_{m,\tau,M}$ is the
space of piecewise polynomial function that have a discontinuity or a
continuity until $C^{m-1}$ in the junctions of the intervals,
determined by the values of the multiplicity's
vector
$$
M=\{m_1,\dots,m_{l-1}\},\quad m_i\in\mN,\quad 1\leq m_i\leq m+1.
$$
Formally:
\begin{multline*}
  S_{m,\tau,M}=\{f:[a,b]\rightarrow\mR\ \mid\ \exists p_0\dots
  p_{l-1}\in\Pi_m \ \text{such that}\\
  f(t)=p(t),\ \forall t\in I_i,\
  i=0\dots l-1\ \text{and}\\
  p_{i-1}^{(j)}(\tau_i)=p_{i}^{(j)}(\tau_i),\ j=0,\dots,m-m_i,\ i=1,\dots,l-1\}.
\end{multline*}
The dimension of the space is between the previous two, and is true
that:
$$
\Pi_m\subseteq S_{m,\tau}\subseteq S_{m,\tau,M}\subseteq P_{m,\tau},
$$
in fact:
\begin{itemize}
  \item if $m_i=1$ for all $i=1,\dots,l-1$, then
    $S_{m,\tau,M}=S_{m,\tau}$;
  \item if $m_i=m+1$ for all $i=1,\dots,l-1$, then
    $S_{m,\tau,M}=P_{m,\tau}$.
\end{itemize}

\subsection{Truncated-powers base for classic splines}\label{sec:truncpow}
A truncated power $(t-\tau_i)_+^m$ is
defined by
$$
(t-\tau_i)_+^m=
\begin{cases}
  0,&\mbox{if}\quad t\leq\tau_i\\
  (t-\tau_i)^m, &\mbox{otherwise}.
\end{cases}
$$
Is possible to demonstrate that the functions
$$
g_i(t)=(t-\tau_i)_+^m)\ \in S_{m\tau},\quad i=1,\dots,l-1
$$
are linearly independents, and that
$$
1,t,t^2,\dots,t^m,(t-\tau_1)_+^m,\dots,(t-\tau_{l-1})_+^m
$$
form a base for the classic spline functions space. A generic element
from this space can be expressed like
$$
\mathbf{S}(t)=\sum_{i=0}^m \mathbf{c_i}\cdot t^i\, +\, \sum_{j=1}^{l-1} \mathbf{d_i}\cdot (t-\tau_j)_+^m
$$
but this form is not a practical representation of a spline because
there isn't an intuitive correlation between the points
$\mathbf{c_i}$, $\mathbf{d_j}$ and the curve. For this purpose we
define the \emph{B-splines base} in section~\ref{sec:bsplines}.

\subsection{B-Splines base for classic splines}\label{sec:bsplines}
\emph{B-splines} are splines defined with a specific base. In this
paragraph we consider only the classic splines $S_{m,\tau}$ and not
the generalized splines $S_{m,\tau,M}$, furthermore we consider the
\emph{order}: $k=m+1$.

For defining the B-splines we need to extend the partition vector
$\tau=\{\tau_0,\cdots,\tau_l\}$ with $k-1$ nodes to the left and $k-1$ to
the right, so we define a new vector
$$
T=\{t_0,\dots,t_{k-2},t_{k-1},\dots,t_{n+1},t_{n+2},\dots,t_{n+k}\}
$$
such that
$$
t_0\leq\dots\leq t_{k-2}\leq \{t_{k-1}\equiv\tau_0\equiv a\}<\dots<
\{t_{n+1}\equiv\tau_l\equiv b\}\leq t_{n+2}\leq\dots\leq t_{n+k}.
$$
$\tau$ have $l+1$ elements, so we can calculate the value of
$$
n=l+k-2,
$$
and the dimension of $S_{m,\tau}$ is
$$
l+m=l+k-1=n+1
$$
that is the number of necessary bases for the space.

The $n+1$ basis $N_{i,k}(t)$ of the B-splines of order $k$ for
$i=0,\dots,n$ are defined by the 
recursive formula:
\begin{align*}
  N_{i,1}(t) &=
  \begin{cases}
    1,\quad \mbox{if}\quad t_i\leq t<t_{i+1}\\
    0,\quad \mbox{otherwise}
  \end{cases}\\
  N_{i,k}(t) &= \omega_{i,k-1}(t)\cdot N_{i,k-1}(t)\ +\
  (1-\omega_{i+1,k-1}(t))\cdot N_{i+1,k-1}(t)
\end{align*}
where
$$
\omega_{i,k}(t) = \frac{t-t_i}{t_{i+k}-t_i}.
$$

The elements of the classic splines space can be expressed in the form
$$
\mathbf{S}(t)=\sum_{i=0}^n\mathbf{v_i}\cdot N_{i,k}(t),
$$
and this representation is more convenient respect to the one of
section~\ref{sec:truncpow} because the curve $\mathbf{S}(t)$ roughly
follow the shape given by the points $\mathbf{v_i}$. Those points are
called \emph{control vertexes} and the polygon defined by them is
called \emph{control polygon} and they can be used to control the
shape of the curve.
\end{document}

%%% Local Variables:
%%% mode: latex
%%% TeX-master: t
%%% End:
