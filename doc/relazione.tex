\documentclass[a4paper,oneside]{article}
\usepackage[utf8]{inputenc}
\usepackage{lmodern}
\usepackage[T1]{fontenc}
\usepackage[italian]{babel}
\usepackage{microtype}
\usepackage{acronym}
\usepackage{mathtools}
\usepackage{braket}
\usepackage{amsfonts}
\usepackage[hidelinks,breaklinks=true]{hyperref}
\usepackage{xcolor}
\usepackage{listings}

\newcommand{\mr}{\ensuremath{\mathbf{R}}}
\newcommand{\me}{\ensuremath{\mathrm{e}}}
\newcommand{\md}{\ensuremath{\mathrm{d}}}
\newcommand{\expected}[1]{\ensuremath{\mathrm{\textbf{E}}\left[#1\right]}}
\newcommand{\variance}[1]{\ensuremath{\mathrm{\textbf{Var}}\left(#1\right)}}
\newcommand{\prob}[1]{\ensuremath{\mathrm{\textbf{P}}\left(#1\right)}}
\newcommand{\abs}[1]{\ensuremath{\left|#1\right|}}
\newcommand{\codei}[1]{\texttt{#1}}

\lstdefinestyle{customPy}{
 language=python,
 showstringspaces=false,
 basicstyle=\footnotesize\ttfamily,
 keywordstyle=\bfseries\color{green!40!black},
 commentstyle=\itshape\color{purple!40!black},
 identifierstyle=\color{blue},
 stringstyle=\color{orange},
}

%\DeclarePairedDelimiter\abs{\lvert}{\rvert}

\author{
  {\Large Stefano Martina}\\
  {\small stefano.martina@stud.unifi.it}\\
  Universit\`a degli Studi di Firenze\\
  Scuola di Scienze Matematiche, Fisiche e Naturali\\
  Corso magistrale di Informatica
}
\title{{\Huge\bfseries Path planning con diagrammi di Voronoi}\\{\large\bfseries
    Esame di Metodi Numerici per la Grafica}}

\begin{document}
\maketitle
\thispagestyle{empty}
\vfill
\begin{abstract}
  Questo lavoro presenta una breve introduzione ai \acf{VD} e al loro
  uso nel \acf{PP}.
\end{abstract}
\clearpage
\acresetall

\section*{\aclp{DV}}
I \acfp{DV} sono definiti a partire da un set di $n$ punti $p_i$ nello spazio,
intuitivamente \`e un set di $n$ regioni $V(p_i)$, tante quanti i punti
iniziali, che partizionano lo spazio in
modo che ogni regione $V(p_i)$ definisce lo spazio dei punti che sono
pi\`u vicini a $p_i$ rispetto ad ogni altro $p_j$ con $j\ne i$.

Pi\`u formalmente si indica con $P=\{p_1,p_2,\dots,p_n\}\in\mr^2$ il
set di punti iniziali con coordinate $(x_1, y_1),\dots,(x_n,y_n)$ nel
piano cartesiano. La regione
\begin{eqnarray*}
  V(p_i) &=& \Set{ x\in\mr^2 | d(x,p_i)\le d(x,p_j), \forall p_j\in
    P\setminus\{p_i\}} \\
  &=& \bigcap_{p_j\in P\setminus\{p_i\}}\Set{ x\in\mr^2 | d(x,p_i)\le d(x,p_j)}
\end{eqnarray*}
dove $d(q,v)$ \`e la distanza tra i due punti $q$ e $v$,
indica il \emph{poligono planare di Voronoi} associato a $p_i$. $p_i$
viene chiamato \emph{punto generatore} o semplicemente
\emph{generatore} di $V(p_i)$.

Il set
\begin{eqnarray*}
 V(P) &=& \Set{ V(p_1),\dots,V(p_n)}\\
 &=&\bigcup_{p_i\in P}V(p_i)\\
 &=&\bigcup_{p_i\in P}\left[\bigcap_{p_j\in P\setminus\{p_i\}}\Set{ x\in\mr^2 | d(x,p_i)\le d(x,p_j)}\right]
\end{eqnarray*}
\`e il \emph{\acl{DV} planare} generato da $P$, o
\emph{\acl{DV}}
di $P$. $P$ \`e detto \emph{set generatore} di $V(P)$.

I \acp{DV} sono costituiti esclusivamente da segmenti uniti da
vertici, formano quindi un reticolo e possono essere assimilati ad un
grafo. Valgono le propriet\`a:
\begin{itemize}
\item un punto $q$ \`e un vertice di $V(P)$ se e solo se la pi\`u grande
  circonferenza vuota\footnote{Senza nessun punto $p_i\in P$ tra il
    centro e la circonferenza.} centrata in $q$ ha tre o pi\`u punti
  $\in P$ sulla circonferenza;
\item il numero di vertici di $V(P)$ \`e al massimo pari a $2n-5$;
\item il numero di lati di $V(P)$ \`e al massimo pari a $3n-6$.
\end{itemize}

\section{Uso dei \aclp{DV} nel \acl{PP}}
Il problema del \ac{PP} consiste nel trovare il percorso
ottimale nello spazio
da un certo punto iniziale $p_s$ ad un certo punto finale
$p_e$ scansando ostacoli definiti da punti $p_1,\dots, p_n$, dove nel
caso pi\`u semplice tali punti rappresentano ostacoli puntiformi, nel
caso più complesso questi rappresentano i vertici di ostacoli
poligonali.

Esistono diversi metodi per risolvere il \ac{PP}, raggruppati nelle
categorie:
\begin{itemize}
\item basati sul potenziale;
\item decomposizione di celle;
\item \emph{roadmap}.
\end{itemize}
I \acp{DV} trovano campo nei metodi basati su \emph{roadmap}. In
generale i metodi \emph{roadmap} cercano di rappresentare lo spazio
vuoto con grafi, e in seguito su questo grafo viene eseguito un
algoritmo di ricerca del percorso minimo tra due punti, ad esempio
quello di Dijkstra. I \acp{DV} possono essere usati per costruire
suddetto grafo.

Un metodo alternativo per costruire il grafo dello spazio vuoto \`e
quello del \ac{VG}. Questo consiste nel tracciare un grafo che comprenda tutti i punti
$p_s,p_e,p_1,\dots,p_n$ dove ogni punto \`e collegato a tutti gli
altri punti che sono visibili, ossia non coperti da
ostacoli.
I vantaggi di usare i \acp{DV} rispetto a \ac{VG} risiede nella
velocit\`a di calcolo del \ac{DV}, $O(n \log n)$ rispetto a $O(n^2)$,
e nel fatto che il percorso finale non tocca vertici e lati degli
ostacoli. Viceversa uno svantaggio di usare \ac{VG} \`e che
il percorso finale non \`e esattamente il pi\`u breve perch\'e
potrebbe passare pi\`u vicino agli ostacoli, ma questo pu\`o essere
mitigato applicando processi di raffinamento del percorso.

Nel caso in cui gli ostacoli siano poligonali occorre eliminare i
segmenti del \ac{DV} che attraversano i poligoni, ma questo processo
pu\`o portare in alcuni casi ad eliminare percorsi utili. Il miglior
approccio quindi consiste nell'aggiungere
punti ridondanti distribuiti lungo il perimetro, calcolare il \ac{DV}
su tutti i punti (vertici pi\`u quelli aggiunti) ed eliminare gli
archi del \ac{DV} che attraversano i poligoni.

Un differente approccio per \ac{PP} nel caso di ostacoli poligonali
consiste nell'usare i \acp{DVG}. I \acp{DVG} generalizzano i \acp{DV}
prendendo in considerazione non solo punti ma anche segmenti, quindi
le regioni sono divise con i criteri:
\begin{itemize}
\item nel caso di due punti le due regioni sono divise da una retta
  perpendicolare al segmento che unisce i punti e passante per il
  punto mediano\footnote{Come nei \acp{DV}.};
\item nel caso di un punto ed un segmento le regioni sono divise da una
  parabola in cui il punto \`e il fuoco ed il vertice \`e il punto
  mediano del segmento che unisce il punto e la retta
  perpendicolarmente;
\item nel caso di due segmenti le regioni sono separate dalla retta
  bisettrice dell'angolo formato dal prolungamento dei segmenti\footnote{Se i
  segmenti sono paralleli allora la retta che divide le regioni \`e
  parallela ai segmenti e passa per il punto medio tra i due
  segmenti.}.
\end{itemize}
Quindi il \ac{DVG} di un insieme di poligoni, costituiti da lati e
vertici, \`e costituito dall'unione di segmenti e archi di
parabola. Rispetto al metodo dei punti ridondanti questo \`e pi\`u
preciso.
\section{Calcolo del \acl{DV}}
Per calcolare il \ac{DV} di un insieme di punti viene usato
l'algoritmo di Fortune.

\newpage
\section*{Acronimi}
\addcontentsline{toc}{section}{Acronimi}
\begin{acronym}[DVG]
  \acro{DV}{diagramma di Voronoi}
  \acro{DV}{diagramma di Voronoi generalizzato}
  \acro{PP}{path planning}
  \acro{VG}{visibility graph}
\end{acronym}
\acrodefplural{DV}[DV]{diagrammi di Voronoi}
\acrodefplural{DVG}[DVG]{diagrammi di Voronoi generalizzati}

\nocite{*}
\phantomsection
\addcontentsline{toc}{section}{\refname}
%\bibliographystyle{apalike}
\bibliographystyle{alpha}
\bibliography{relazione}

\end{document} 
